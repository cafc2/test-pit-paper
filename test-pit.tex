\documentclass[conference]{IEEEtran}
\usepackage{amsmath}
\usepackage{microtype}
\usepackage{flushend}

\title{A Statistical Approach to Test Pitting}
\author{%
    \IEEEauthorblockN{Rich Wareham}
    \IEEEauthorblockA{Signal Processing and Communications Laboratory\\
    Department of Engineering, University of Cambridge\\
    Email: rjw57@cam.ac.uk}
}

\begin{document}

\maketitle

\begin{abstract}
    We develop a simple threshold-based test for determining the significance of
    one test pit when compared to others within some shared context, for example
    a single village. We do this by developing a statistical model for test pit
    evolution and then use the model to provide a data-driven test for
    significance based on Welch's $t$ test\cite{welch}. For our purposes,
    `significant' is taken to mean `surprising given the wider context'. Such
    significant test pits can be used to localise settlement or other processes
    leading to larger-scale pottery deposition.
\end{abstract}

\section{Introduction}

In this text we consider a statistical approach to analysing test pit data in
which we assign a number indicating the significance, in a statistical sense,
of a given site given the wider context of the area. We do this by comparing
sherd counts which are partitioned into different eras.

Conceptually we wish to find some statistic which reflects our prior belief that a
test pit which has a larger proportion of, for example, Roman pottery in it in
comparison to other test pits in the area is in some way more `surprising',
i.e.\ has a lower probability, than test pits whose pottery proportions match
the site as a whole. We aim that this statistic be useful for `calling
attention' to interesting test pits.

Throughout this document we shall consider a single test pit within a collection
of pits on the same site and use the count of pottery as the measured value.
The mathematics works for both pottery weight or sherd counts, although shred
counts are more convenient, and may trivially be extended to consider the
probability of entire sites within wider contexts of settlement by considering
an individual site as a single large test pit. Indeed, such an approach is
likely, given the greater quantities of data, to lead to firmer conclusions at
the expense of coarser localisation.

For the purposes of illustration, we will consider pottery to fall into the
categories Roman, Early Medieval, Late Medieval, High Medieval. This
categorisation is purely conventional and the technique below will work with any
pottery categorisation assuming that no one sherd may fall into more than one
category and that all shreds being considered have an associated category.

\section{Notation}

Assume that there are $N$ test pits in the site. For the $i$-th test pit we
denote the count of Roman era pottery sherds as $R_i$, the count of Early
Medieval pottery as $E_i$, Late Medieval as $L_i$ and High Medieval as $H_i$.
The total count of pottery sherds is denoted as $T_i$ and we make the assumption
that all pottery in the pit comes from one of our designated eras. That is to
say that we assume
\[
    T_i = R_i + E_i + L_i + H_i.
\]

Given these individual test pit counts we can derive the \emph{total} count.
For example the total count of Roman pottery, $R$, is given by
\[
    R = R_1 + R_2 + \cdots + R_N \equiv \sum_{i=1}^N R_i
\]
and similar expressions can be written down for the total amounts of Early, Late
and High Medieval, $E$, $L$ and $H$ respectively and also the total count of
pottery over all pits, $T = R + E + L + H$.

Sometimes, for convenience's sake, we refer to the \emph{normalised} counts,
for example $R'_i \equiv R_i / T_i$ which can be interpreted as the
\emph{proportion} of, in this instance, Roman pottery within the pit.

\section{Welch's t test}

If we let $R_{\setminus i}$ and $T_{\setminus i}$ be the Roman era and total
counts \emph{excluding} test pit $i$ then we note that we have two completely
independent estimators for the proportion of Roman pottery on the site:
\[
    R'_i = \frac{R_i}{T_i}, \quad \mbox{and,} \quad
    R'_{\setminus i} = \frac{R_{\setminus i}}{T_{\setminus i}}.
\]

If test pit $i$ is unsurprising, we expect $R'_i$ to approximately equal
$R'_{\setminus i}$. Welch's $t$ test\cite{welch} can be used to determine if any
differences are statistically significant. Specifically, for each test pit, $i$,
one computes the statistic
\[
    t^{(R)}_i = \frac{A^{(i)}_R}{\sqrt{B^{(i)}_R}}
\]
where
\[
    A^{(i)}_R = \frac{R_i}{T_i} - \frac{R_{\setminus i}}{T_{\setminus i}},
    \quad\mbox{and,}\quad
    B^{(i)}_R = \frac{\sigma^2_i}{T_i} +
    \frac{\sigma^2_{\setminus i}}{T_{\setminus i}},
\]
and $\sigma^2_i$ and $\sigma^2_{\setminus i}$ are the sample variances.

For sherd weights, these variances would have to be calculated by measuring the
weight of each individual sherd. For sherd counts we can make use of a trick: we
know the samples for test pit $i$ consist of $R_i$ ones and $T_i-R_i$ zeros,
\[
    \sigma^2_i = \frac{1}{T_i-1} \left[
        R_i\left(1-\frac{R_i}{T_i}\right)^2 +
    (T_i-R_i)\left(-\frac{R_i}{T_i}\right)^2 \right]
\]
by simple application of the formula for sample variance. Multiplying out, we
obtain:
\begin{align*}
    \sigma^2_i &= \frac{1}{T_i-1} \left[
        R_i - \frac{2R_i^2}{T_i} + \frac{R_i^3}{T_i^2} +
        \frac{T_iR_i^2}{T_i^2} - \frac{R_i^3}{T_i^2}
    \right] \\
    &= \frac{1}{T_i-1} \left[
        R_i - \frac{R_i^2}{T_i} 
    \right] =
    \frac{T_i R_i - R_i^2}{T_i(T_i-1)}.
\end{align*}
One can similarly obtain a relation for $\sigma^2_{\setminus i}$ and hence,
\[
    B^{(i)}_R =
    \frac{T_i R_i - R_i^2}{T^2_i(T_i-1)} +
    \frac{T_{\setminus i} R_{\setminus i} - R_{\setminus i}^2}{T^2_{\setminus
    i}(T_{\setminus i}-1)}.
\]

There exists a standard formula\cite{welch} for the degrees of freedom, $\nu^{(R)}_i$, from
which we can obtain a $p$-value for the null hypothesis via the Student's
$t$-distribution which is parameterised by $\nu^{(R)}_i$. For our purposes,
however, it is usually sufficient to make use of the rule of thumb that
\[
    \left|\frac{A^{(i)}_R}{\sqrt{B^{(i)}_R}}\right| \ge 3
\]
indicates a statistically significant result. (For degrees of freedom from 1 to
$+\infty$ this rule is reasonable.) For completeness, the relation for
$\nu^{(R)}_i$ is
\[
    \nu^{(R)}_i = \frac{\left( B^{(i)}_R \right)^2}{C^{(i)}_R}
\]
where
\[
    C^{(i)}_R = \frac{\sigma^4_i}{T_i^2 (1-T_i)} + \frac{\sigma^4_{\setminus
    i}}{T_{\setminus i}^2 (1-T_{\setminus i})}.
\]

\section{Conclusion}

In the section above we obtained a statistic which may be used to evaluate the
significance of one test pit in comparison to others within a similar location.
Specifically, for the example of Roman pottery, the value
\[
    t^{(R)}_i = \frac{R_i \cdot T^{-1}_i - R_{\setminus i} \cdot T^{-1}_{\setminus
    i}}{\sqrt{
    \frac{T_i R_i - R_i^2}{T^2_i(T_i-1)} +
    \frac{T_{\setminus i} R_{\setminus i} - R_{\setminus i}^2}{T^2_{\setminus
    i}(T_{\setminus i}-1)}}}
\]
may be computed for the $i$-th test pit and can be interpreted as a number of
standard deviations away from the null hypothesis. Consequently, a value of
$|t^{(R)}_i| \ge 3$ for test pit $i$ can be used to conclude that the proportion
of Roman pottery in test pit $i$ differs from neighbouring test pits in a
statistically significant way. The sign of $t^{(R)}_i$ indicates whether this
differences is in the form of a surprisingly little (-ve) or a surprisingly
large (+ve) amount of pottery. Similar expressions can be formed for the other
eras.

It should be noted that this statistic is merely a value which can be used to
directly compare test pits based on the assumptions stated above. Specifically,
we have assumed that test pits which contain a relatively large amount of
pottery from one era or a relatively large amount of pottery in general are more
indicative of the ``true'' pottery distribution and that such indicative test
pits which vary greatly from the rest of the site are significant. The aim of
the statistic is to flag those test pits worthy of closer inspection. Any
conclusions which may be drawn from those pits are better left to an arch\ae
ologist directly inspecting the nature of the sherds.

\begin{thebibliography}{1}
\bibitem{welch}
    Welch, B.~L.~(1947). \emph{The generalization of ''Student's`` problem when
    several different population variances are involved}. Biometrika \textbf{34}
    (1--2): 28--35. doi:10.1093/biomet/34.1-2.28. MR 1.
\end{thebibliography}

\end{document}

% vim:sw=4:sts=4:et:tw=80
